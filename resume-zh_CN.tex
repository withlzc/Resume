% !TEX TS-program = xelatex
% !TEX encoding = UTF-8 Unicode
% !Mode:: "TeX:UTF-8"

\documentclass{resume}
\usepackage{zh_CN-Adobefonts_external} % Simplified Chinese Support using external fonts (./fonts/zh_CN-Adobe/)
% \usepackage{NotoSansSC_external}
% \usepackage{NotoSerifCJKsc_external}
% \usepackage{zh_CN-Adobefonts_internal} % Simplified Chinese Support using system fonts
\usepackage{linespacing_fix} % disable extra space before next section
\usepackage{cite}

\begin{document}
\pagenumbering{gobble} % suppress displaying page number

\name{雷志成}

% \basicInfo{
  \centerline {\phone{+86 18501969152} \textperiodcentered\ 
  \email{withlzc@163.com}}
  % \github[withlzc]{https://github.com/withlzc}
  % \linkedin[billryan8]{https://www.linkedin.com/in/billryan8}
% }

\section{\faUsers\ 个人简介}
\begin{itemize}
  \item 一年Golang后端开发实习经验,熟悉项目开发流程以及自动化部署
  \item 熟悉HTTP、TCP/IP、WebSocket等网络协议,了解PostgreSQL、MySQL、Redis等数据库
  \item 熟悉Linux开发环境
\end{itemize}

\section{\faGraduationCap\ 教育背景}
\datedsubsection{\textbf{北京邮电大学}}{2018.09 -- 至今}
\textit{硕士}\ 软件工程
\\核心课程:网络软件设计,软件测试技术,数据库系统设计与开发,高级操作系统,需求工程,数据仓库与知识发现
\\平均绩点:3.42 / 4

\datedsubsection{\textbf{西安工程大学}}{2014.09 -- 2018.07}
\textit{本科}\ 信息与计算科学
\\核心课程:高级语言程序设计,数据结构与算法,C++程序设计,Java与面向对象程序设计,计算机网络,数据库原理及应用
\\平均绩点:2.58 / 4

% 项目经历:产品、技术、结果
% 3句话,你做了什么,你用了什么技术,有哪些impact(给数据,比如traffic/users 提升了多少,delay 少了多少,处理了多大的文件)

\section{\faBriefcase\ 实习/项目经历}
\datedsubsection{\textbf{灵汐科技},北京}{2019.09 -- 2020.06}
\role{Golang}{后端开发实习}
\textbf{智能安防监控系统}
\begin{itemize}
  % \item 开发系统数据库缓存模块,提高数据访问速度
  \item 开发系统业务日志中间件,配合Swagger生成的API文档自动生成业务日志
  % \item 优化系统消息通知模块,使用WebSocket实时推送
  \item 通过封装WebSocket实现系统消息通知模块,提升消息通知的实时性与稳定性
  \item 修改后端推送摄像头实时视频流的方式,前端使用http-flv播放
  \item 重构人脸比对任务的worker以及pipeline
  % \item 使用Docker部署系统,开发脚本优化部署流程
\end{itemize}

\textbf{智能巡检系统}
\begin{itemize}
  \item 完成业务管理平台的概要设计与详细设计
  \item 开发业务管理平台中设备管理、预置位管理、任务管理等多个模块
  % \item 优化系统设计,尽可能实现模块间解耦
  \item 开发采集与分析模块,通过cgo调用动态库与摄像头、机械臂以及算法模块交互
\end{itemize}

\textbf{单机多卡调度系统}
\begin{itemize}
  \item 设计共享状态调度的总体方案以及架构
  \item 开发任务池、调度器、资源管理器、节点管理等模块
  \item 解决调度过程中产生的资源冲突、故障转移等问题
\end{itemize}

\datedsubsection{\textbf{北邮人论坛搜索引擎}}{2019.05 -- 2019.06}
\role{Python}{课程项目}
\begin{onehalfspacing}
面向北邮人论坛所有帖子的全局搜索引擎
\begin{itemize}
  \item 使用Scrapy框架开发爬虫模块获取论坛所有帖子信息存入ElasticSearch
  % \item 将数据存入ElasticSearch,建立索引
  \item 后台使用Diango框架,从ElasticSearch搜索帖子信息并与前端交互
\end{itemize}
\end{onehalfspacing}

% \section{\faUsers\ 项目经历}
% \role{Python}{课程项目}
% \begin{onehalfspacing}
% 面向北邮人论坛所有帖子的全局搜索引擎
% \begin{itemize}
%   \item 使用Scrapy框架开发爬虫模块获取论坛所有帖子信息存入ElasticSearch
%   % \item 将数据存入ElasticSearch,建立索引
%   \item 后台使用Diango框架,从ElasticSearch搜索帖子信息并与前端交互
% \end{itemize}
% \end{onehalfspacing}

\section{\faCogs\ 技能}
% increase linespacing [parsep=0.5ex]
\begin{itemize}[parsep=0.5ex]
  \item 开发语言: Golang、Java、C++、Python、Shell
  % \item 平台: Linux
  % \item 框架: Gin、SpringBoot、Django
  \item 工具: Git、Docker
  \item 语言: 英语 - 熟练 (CET 6)
\end{itemize}

% \section{\faHeartO\ 获奖情况}
% \datedline{2016-2017学年综合素质测评奖学金}{2017年5月}
% \datedline{校级优秀毕业生}{2018年6月}

% \section{\faInfo\ 其他}
% % increase linespacing [parsep=0.5ex]
% \begin{itemize}[parsep=0.5ex]
%   % \item 博客: https://withlzc.github.io
%   % \item GitHub: https://github.com/withlzc
%   \item 语言: 英语 - 熟练 (CET 6)
% \end{itemize}

%% Reference
%\newpage
%\bibliographystyle{IEEETran}
%\bibliography{mycite}
\end{document}
