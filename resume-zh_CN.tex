% !TEX TS-program = xelatex
% !TEX encoding = UTF-8 Unicode
% !Mode:: "TeX:UTF-8"

\documentclass{resume}
\usepackage{zh_CN-Adobefonts_external} % Simplified Chinese Support using external fonts (./fonts/zh_CN-Adobe/)
% \usepackage{NotoSansSC_external}
% \usepackage{NotoSerifCJKsc_external}
% \usepackage{zh_CN-Adobefonts_internal} % Simplified Chinese Support using system fonts
\usepackage{linespacing_fix} % disable extra space before next section
\usepackage{cite}

\begin{document}
\pagenumbering{gobble} % suppress displaying page number

\name{雷志成}

\basicInfo{
  \phone{(+86) 18501969152} \textperiodcentered\ 
  \email{withlzc@163.com} \textperiodcentered\ 
  \github[withlzc]{https://github.com/withlzc}
  % \linkedin[billryan8]{https://www.linkedin.com/in/billryan8}
  }
 
\section{\faGraduationCap\  教育背景}
\datedsubsection{\textbf{北京邮电大学}, 北京}{2018 -- 至今}
\textit{硕士}\ 软件工程
\\核心课程:网络软件设计,软件测试技术,数据库系统设计与开发,需求工程,数据仓库与知识发现
\datedsubsection{\textbf{西安工程大学}, 西安, 陕西}{2014 -- 2018}
\textit{本科}\ 信息与计算科学
\\核心课程:高级语言程序设计,数据结构与算法,C++程序设计,Java与面向对象程序设计,计算机网络,数据库原理及应用

\section{\faUsers\ 项目经历}
% \datedsubsection{\textbf{黑科技公司} 上海}{2015年3月 -- 2015年5月}
% \role{实习}{经理: 高富帅}
% xxx后端开发
% \begin{itemize}
%   \item 实现了 xxx 特性
%   \item 后台资源占用率减少8\%
%   \item xxx
% \end{itemize}

% \datedsubsection{\textbf{分布式科学上网姿势}}{2014年6月 -- 至今}
% \role{Golang, Linux}{个人项目,和富帅糕合作开发}
% \begin{onehalfspacing}
% 分布式负载均衡科学上网姿势, https://github.com/cyfdecyf/cow
% \begin{itemize}
%   \item 修复了连接未正常关闭导致文件描述符耗尽的 bug
%   \item 使用Chord 哈希 URL, 实现稳定可靠地分流
%   \item xxx (尽量使用量化的客观结果)
% \end{itemize}
% \end{onehalfspacing}

% 3句话,你做了什么,你用了什么技术,有哪些impact(给数据,比如traffic/users 提升了多少,delay 少了多少,处理了多大的文件)
\datedsubsection{\textbf{个人博客及管理系统}}{2019年7月 -- 2019年8月}
\role{SpringBoot + Vue}{个人项目}
\begin{onehalfspacing}
简易个人博客以及后台博客管理系统
\begin{itemize}
  \item 使用SpringBoot和Vue进行前后端分离开发
  \item 使用MySQL作为数据库存储相关数据
  \item 实现了博客前端展示以及后台对博客以及页面的管理
\end{itemize}
\end{onehalfspacing}

\datedsubsection{\textbf{北邮人论坛搜索引擎}}{2019年5月 -- 2019年6月}
\role{Python}{个人项目}
\begin{onehalfspacing}
面向北邮人论坛所有帖子的全局搜索引擎
\begin{itemize}
  \item 使用Scrapy框架爬取论坛所有帖子信息
  \item 将数据存入ElasticSearch,建立索引
  \item 搜索引擎后台使用Diango框架,从ElasticSearch搜索并与前端交互
\end{itemize}
\end{onehalfspacing}

% Reference Test
%\datedsubsection{\textbf{Paper Title\cite{zaharia2012resilient}}}{May. 2015}
%An xxx optimized for xxx\cite{verma2015large}
%\begin{itemize}
%  \item main contribution
%\end{itemize}

\section{\faCogs\ IT 技能}
% increase linespacing [parsep=0.5ex]
\begin{itemize}[parsep=0.5ex]
  \item 编程语言: Java、C++、C、Python、HTML、CSS、JavaScript
  % \item 平台: Linux
  \item 框架: Spring、SpringBoot、Django、Vue
\end{itemize}

\section{\faHeartO\ 获奖情况}
\datedline{2016-2017学年综合素质测评奖学金}{2017年5月}
\datedline{校级优秀毕业生}{2018年6月}

\section{\faInfo\ 其他}
% increase linespacing [parsep=0.5ex]
\begin{itemize}[parsep=0.5ex]
  % \item 博客: https://withlzc.github.io
  % \item GitHub: https://github.com/withlzc
  \item 语言: 英语 - CET 6
\end{itemize}

%% Reference
%\newpage
%\bibliographystyle{IEEETran}
%\bibliography{mycite}
\end{document}
